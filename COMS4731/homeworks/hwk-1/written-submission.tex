%%%%%%%%%%%%%%%%%%%%%%%%%%%%%%%%%%%%%%%%%
% Programming/Coding Assignment
% LaTeX Template
%
% This template has been downloaded from:
% http://www.latextemplates.com
%
% Original author:
% Ted Pavlic (http://www.tedpavlic.com)
%
% Note:
% The \lipsum[#] commands throughout this template generate dummy text
% to fill the template out. These commands should all be removed when 
% writing assignment content.
%
% This template uses a Perl script as an example snippet of code, most other
% languages are also usable. Configure them in the "CODE INCLUSION 
% CONFIGURATION" section.
%
%%%%%%%%%%%%%%%%%%%%%%%%%%%%%%%%%%%%%%%%%

%----------------------------------------------------------------------------------------
%	PACKAGES AND OTHER DOCUMENT CONFIGURATIONS
%----------------------------------------------------------------------------------------

\documentclass{article}

\usepackage{fancyhdr} % Required for custom headers
\usepackage{extramarks} % Required for headers and footers
%\usepackage[usenames,dvipsnames]{color} % Required for custom colors
%\usepackage{graphicx} % Required to insert images
%\usepackage{listings} % Required for insertion of code
%\usepackage{courier} % Required for the courier font

% Margins
\topmargin=-0.45in
\evensidemargin=0in
\oddsidemargin=0in
\textwidth=6.5in
\textheight=9.0in
\headsep=0.25in

\linespread{1.1} % Line spacing

% Set up the header and footer
\pagestyle{fancy}
\lhead{\hmwkAuthorName} % Top left header
\chead{\hmwkClass: \hmwkTitle} % Top center head
\rhead{} % Top right header
\lfoot{\lastxmark} % Bottom left footer
\cfoot{} % Bottom center footer
\rfoot{} % Bottom right footer
\renewcommand\headrulewidth{0.4pt} % Size of the header rule
\renewcommand\footrulewidth{0.4pt} % Size of the footer rule

\setlength\parindent{0pt} % Removes all indentation from paragraphs

%%----------------------------------------------------------------------------------------
%%	CODE INCLUSION CONFIGURATION
%%----------------------------------------------------------------------------------------
%
%\definecolor{MyDarkGreen}{rgb}{0.0,0.4,0.0} % This is the color used for comments
%\lstloadlanguages{Perl} % Load Perl syntax for listings, for a list of other languages supported see: ftp://ftp.tex.ac.uk/tex-archive/macros/latex/contrib/listings/listings.pdf
%\lstset{language=Perl, % Use Perl in this example
%        frame=single, % Single frame around code
%        basicstyle=\small\ttfamily, % Use small true type font
%        keywordstyle=[1]\color{Blue}\bf, % Perl functions bold and blue
%        keywordstyle=[2]\color{Purple}, % Perl function arguments purple
%        keywordstyle=[3]\color{Blue}\underbar, % Custom functions underlined and blue
%        identifierstyle=, % Nothing special about identifiers                                         
%        commentstyle=\usefont{T1}{pcr}{m}{sl}\color{MyDarkGreen}\small, % Comments small dark green courier font
%        stringstyle=\color{Purple}, % Strings are purple
%        showstringspaces=false, % Don't put marks in string spaces
%        tabsize=5, % 5 spaces per tab
%        %
%        % Put standard Perl functions not included in the default language here
%        morekeywords={rand},
%        %
%        % Put Perl function parameters here
%        morekeywords=[2]{on, off, interp},
%        %
%        % Put user defined functions here
%        morekeywords=[3]{test},
%       	%
%        morecomment=[l][\color{Blue}]{...}, % Line continuation (...) like blue comment
%        numbers=left, % Line numbers on left
%        firstnumber=1, % Line numbers start with line 1
%        numberstyle=\tiny\color{Blue}, % Line numbers are blue and small
%        stepnumber=5 % Line numbers go in steps of 5
%}
%
%% Creates a new command to include a perl script, the first parameter is the filename of the script (without .pl), the second parameter is the caption
%\newcommand{\perlscript}[2]{
%\begin{itemize}
%\item[]\lstinputlisting[caption=#2,label=#1]{#1.pl}
%\end{itemize}
%}

%----------------------------------------------------------------------------------------
%	DOCUMENT STRUCTURE COMMANDS
%	Skip this unless you know what you're doing
%----------------------------------------------------------------------------------------

% Header and footer for when a page split occurs within a problem environment
\newcommand{\enterProblemHeader}[1]{
\nobreak\extramarks{#1}{#1 continued on next page\ldots}\nobreak
\nobreak\extramarks{#1 (continued)}{#1 continued on next page\ldots}\nobreak
}

% Header and footer for when a page split occurs between problem environments
\newcommand{\exitProblemHeader}[1]{
\nobreak\extramarks{#1 (continued)}{#1 continued on next page\ldots}\nobreak
\nobreak\extramarks{#1}{}\nobreak
}

\setcounter{secnumdepth}{0} % Removes default section numbers
\newcounter{homeworkProblemCounter} % Creates a counter to keep track of the number of problems

\newcommand{\homeworkProblemName}{}
\newenvironment{homeworkProblem}[1][Problem \arabic{homeworkProblemCounter}]{ % Makes a new environment called homeworkProblem which takes 1 argument (custom name) but the default is "Problem #"
\stepcounter{homeworkProblemCounter} % Increase counter for number of problems
\renewcommand{\homeworkProblemName}{#1} % Assign \homeworkProblemName the name of the problem
\section{\homeworkProblemName} % Make a section in the document with the custom problem count
\enterProblemHeader{\homeworkProblemName} % Header and footer within the environment
}{
\exitProblemHeader{\homeworkProblemName} % Header and footer after the environment
}

\newcommand{\problemAnswer}[1]{ % Defines the problem answer command with the content as the only argument
\noindent\framebox[\columnwidth][c]{\begin{minipage}{0.98\columnwidth}#1\end{minipage}} % Makes the box around the problem answer and puts the content inside
}

\newcommand{\homeworkSectionName}{}
\newenvironment{homeworkSection}[1]{ % New environment for sections within homework problems, takes 1 argument - the name of the section
\renewcommand{\homeworkSectionName}{#1} % Assign \homeworkSectionName to the name of the section from the environment argument
\subsection{\homeworkSectionName} % Make a subsection with the custom name of the subsection
\enterProblemHeader{\homeworkProblemName\ [\homeworkSectionName]} % Header and footer within the environment
}{
\enterProblemHeader{\homeworkProblemName} % Header and footer after the environment
}

%----------------------------------------------------------------------------------------
%	NAME AND CLASS SECTION
%----------------------------------------------------------------------------------------

\newcommand{\hmwkTitle}{Assignment\ \#1, Written Portion} % Assignment title
\newcommand{\hmwkDueDate}{2016-09-20} % Due date
\newcommand{\hmwkClass}{COMS\ 4731} % Course/class
\newcommand{\hmwkClassTime}{10:10am} % Class/lecture time
\newcommand{\hmwkClassInstructor}{Shree K Nayar} % Teacher/lecturer
\newcommand{\hmwkAuthorName}{Adam Hadar, anh2130} % Your name

%----------------------------------------------------------------------------------------
%	TITLE PAGE
%----------------------------------------------------------------------------------------

\title{
\vspace{2in}
\textmd{\textbf{\hmwkClass:\ \hmwkTitle}}\\
\normalsize\vspace{0.1in}\small{Due\ on\ \hmwkDueDate}\\
\vspace{0.1in}\large{\textit{\hmwkClassInstructor\ \hmwkClassTime}}
\vspace{3in}
}

\author{\textbf{\hmwkAuthorName}}
\date{} % Insert date here if you want it to appear below your name

%----------------------------------------------------------------------------------------

\begin{document}

\maketitle
\newpage

%----------------------------------------------------------------------------------------
%	PROBLEM 1
%----------------------------------------------------------------------------------------

\begin{homeworkProblem}
Considering a pinhole camera with perspective projection.

\problemAnswer{
    The shape of the image of a disk that is sitting parallel to the image plane will always be the shape of the original disk.

    Given the equations \(\frac{x_i}{f}=\frac{x_o}{z_o}\) and \(\frac{y_i}{f}=\frac{y_o}{z_o}\),

    And given three points on the image disk (that are both therefore the same distance \(z_o\) from the pinhole, \(r_{o1}=(x_{o1},y_{o1},z_o)\), \(r_{o2}=(x_{o2},y_{o2},z_o)\), and \(r_{o3}=(x_{o3},y_{o3},z_o)\),

    You can solve for their positions on the image plane:
    \[x_{i1} = f*x_{o1}/z_o , y_{i1} = f*y_{o1}/z_o\]
    \[x_{i2} = f*x_{o2}/z_o , y_{i2} = f*y_{o2}/z_o\]
    \[x_{i3} = f*x_{o3}/z_o , y_{i3} = f*y_{o3}/z_o\]

    Now, if you try to compute the distances between the object points, and the distances between the image points,
    \[(x_{o2},y_{o2},z_o) - (x_{o1},y_{o1},z_o) = (x_{o2}-x_{o1},y_{o2}-y_{o1},0)\]
    \[(x_{o3},y_{o3},z_o) - (x_{o1},y_{o1},z_o) = (x_{o3}-x_{o1},y_{o3}-y_{o1},0)\]
    \[(x_{o3},y_{o3},z_o) - (x_{o2},y_{o2},z_o) = (x_{o3}-x_{o2},y_{o3}-y_{o2},0)\]
    \[(f*x_{o2}/z_o,f*y_{o2},z_o) - (f*x_{o1}/z_o,f*y_{o1}/z_o,z_o) = ((x_{o2}-x_{o1})*f/z_o,(y_{o2}-y_{o1})*f/z_o,0)\]
    \[(f*x_{o3}/z_o,f*y_{o3},z_o) - (f*x_{o1}/z_o,f*y_{o1}/z_o,z_o) = ((x_{o3}-x_{o1})*f/z_o,(y_{o3}-y_{o1})*f/z_o,0)\]
    \[(f*x_{o3}/z_o,f*y_{o3},z_o) - (f*x_{o2}/z_o,f*y_{o2}/z_o,z_o) = ((x_{o3}-x_{o2})*f/z_o,(y_{o3}-y_{o2})*f/z_o,0)\]
    
    You will find the relationships between those distances are the same between the image ones and the object ones, besides the scaling by a constant \(f/z_o\). Which therefore proves that the relative positions of points on a plane parallel to the image plane will remain constant, while their scale may change.
    }
\end{homeworkProblem}
\begin{homeworkProblem}
\problemAnswer{
    Since the area (\(A\)) is \(1 mm^2\), it's radius must be \(\sqrt{\frac{A}{\pi}}\). From now on we will call this number \(\phi\).
		
    The diameter is therefore \(2 * \phi\).
		
    With an arbitrary value \(Q\), and all units in mm, we can define:
		
    Two points equidistant from each other on the circumference (marking the diameter) of this image would then be \((Q,Q)\) and \((Q,Q+2 * \phi)\).
		
    Knowing the relations \(\frac{x_i}{f} = \frac{x_o}{z_o}\) and \(\frac{y_i}{f} = \frac{y_o}{z_o}\), where subscript \(i\) stand for image dimensions, and subscript \(o\) stand for object dimensions,
		
    And knowing that \(z_o = 1 m = 1000 mm\)

	We can compute the same positions along the circumference for the actual disk:
	\[\frac{Q}{f} = \frac{x_{o1}}{z_o} \to x_{o1} = \frac{Q*z_o}{f}\]
	\[\frac{Q}{f} = \frac{y_{o1}}{z_o} \to y_{o1} = \frac{Q*z_o}{f}\]
	\[\frac{Q}{f} = \frac{x_{o2}}{z_o} \to x_{o2} = \frac{Q*z_o}{f}\]
	\[\frac{Q+2*\phi}{f} = \frac{y_{o2}}{z_o} \to y_{o2} = \frac{(Q+2*\phi)*z_o}{f}\]
	And find them to be: \((\frac{Q*z_o}{f},\frac{Q*z_o}{f})\) and \((\frac{Q*z_o}{f},\frac{(Q+2*\phi)*z_o}{f})\).
		
	If then zo is doubled, we can reverse the process to find the corresponding points on the new image.
	\[\frac{x_{i1}}{f} = \frac{Q*2*\frac{z_o}{f}}{2*z_o} \to x_{i1} = Q\]
	\[\frac{y_{i1}}{f} = \frac{Q*2*\frac{z_o}{f}}{2*z_o} \to y_{i1} = Q\]
	\[\frac{x_{i2}}{f} = \frac{Q*2*\frac{z_o}{f}}{2*z_o} \to x_{i2} = Q\]
	\[\frac{y_{i2}}{f} = \frac{(Q+2*\phi)*2*\frac{z_o}{f}}{2*z_o} \to y_{i2} = Q+2*\phi\]
		
	The new image points are \((Q,Q)\) and \((Q,Q+2*\phi)\).
		
    The diameter of the new circle is the vector \(|(Q,Q)-(Q,Q+2*\phi)| = (0,2*\phi)\). The vector's length can then be computed as \(2*\phi\).
		
    The radius of the new circle is \(\phi\).
		
    The area of the new circle \(= \pi * \phi^2 = \pi * \sqrt{\frac{A}{\pi}}^2 = \pi * \frac{A}{\pi} = A\).
		
    The are of the new circle is the same as the area of the old circle.
}
\end{homeworkProblem}
\begin{homeworkProblem}
\problemAnswer{
    The shape of the image of the sphere will be highly dependent on how distant \(x_o\) and \(y_o\) are from \(0\) (\(0\) being the origin of the system, and therefore the position of the pinhole. Note that \(z_o\) would still be nonzero).
    
    The relationship between points on a plane from part 1.a still hold up. But since the object is now a sphere, there are a large number of new points on the object that are sitting between the original object plane and the pinhole. as the sphere moves further from the origin along that parallel plane, points near the front on the sphere will obstruct the projection of the points of the sphere that are along the parallel plane.
    
    The image that would originally have been a perfect circle when the sphere was sitting at the origin, would stretch out into an ellipsis as it moves further away, as the other points of the sphere are being projected on top of the ones on a given parallel plane. It is as if there is an infinite number of other disks sitting on parallel planes that are closer to the pinhole than the original one.
}
\end{homeworkProblem}

%----------------------------------------------------------------------------------------
%	PROBLEM 2
%----------------------------------------------------------------------------------------

% To have just one problem per page, simply put a \clearpage after each problem

\begin{homeworkProblem}
The prompt requests the we derive the equation for the hyperfocal distance 'H' of an imaging system.
\problemAnswer{
	We know the following equation is true, and represents part of the Depth of Field of a system:

	\[c = \frac{f^{2} \times (o_2 - o)}{N \times o_2 \times (o - f)}\]

    Where \(c\) is the blur circle diameter, \(f\) is the focal length, \(o_2\) is the farthest distance in a system where objects are in focus, and \(N\) is f-number of the lens in the system.

    We can identify \(o\) as equivalent to \(H\), and will therefore restructure our equation with \(H\) on one side:

    \[H = \frac{f^2}{N*c} \times \frac{o_2 - H}{o_2} + f\]

    We know that in a hyperfocal system \(o_2\) must be an infinite value. To better understand how it can be represented in the function, let's take its limit of the function as \(o_2\) approaches \(\infty\):

    \[\lim_{o_2 \to \infty} \frac{o_2 - H}{o_2} = 1\]

    Therefore, our equation for \(H\) can be simplified to:

    \[H = \frac{f^2}{N \times c} + f\]

    Which we can identify as the equation we needed to prove.

    Given the general equation for the Depth of Field of an imaging system, and given that the farthest distance in focus in a hyperfocal system is infinite, we have deduced the specific equation for finding the hyperfocal distance in an imaging system.
}

\end{homeworkProblem}

%----------------------------------------------------------------------------------------
%	PROBLEM 3
%----------------------------------------------------------------------------------------

% To have just one problem per page, simply put a \clearpage after each problem

\begin{homeworkProblem}
\problemAnswer{
	This is a simple implementation of the depth of field equation,
    \[c = \frac{f^2*(o - o_1)}{N*o_1*(o - f)}\]

    where the focal length, the f-number, and the maximized blur circle are all constants of the given image system.

    
}
\end{homeworkProblem}
\begin{homeworkProblem}
\problemAnswer{
    Any given point on the scene line would have been represented as \((x_o,k)\), where because we are operating in only two dimensions, the equation \(\frac{x_o}{f} = \frac{x_i}{k}\) exists (in the 3D plane we knew of two equations like this, but since we dropped a degree of freedom there is now only one equation.

    But now since the scene line has been placed at an angle \(\theta\) from the perpendicular to the optical axis, any given point on the new scene line should actually be represented as \((x_o*cos(\theta),k + x_o*sin(\theta))\).

    This makes intuitive sense - if the new angle is 0, all the points new values should be equal to their old ones. If the angle is 180 then the line has been flipped and all x values should be negated, but their relative positions from the lens should stay the same. If the new angle is 90, then it is as if all the points are at the optical axis (in the x direction), and any y value has added to it the original position of the line in the x direction.

    Since the system is in two dimensions, the line will remain a line regardless of a change in angle because any modifications to the line in this respect keeps all points still adjacent to their neighbors, and no points go into an imaginary axis. However, the image of the line will appear to shrink, as more and more points on the line approach the optical axis.
}
\end{homeworkProblem}
\begin{homeworkProblem}
\problemAnswer{
    
}
\end{homeworkProblem}

%----------------------------------------------------------------------------------------

\end{document}
